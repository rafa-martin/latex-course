\documentclass[a4paper,11pt]{article}

%% Packages
\usepackage[main=spanish,english]{babel} % Language
\usepackage[utf8]{inputenc} % Encoding
\usepackage{lipsum}         % Dummy text
\usepackage{amsmath}        % Math
\usepackage{amssymb}        % Math

\begin{document}

% Title
\title{My first document}
\author{Rafael Martín}
\date{\today}
\maketitle

\tableofcontents

\begin{abstract}
\lipsum[2]
\end{abstract}

\section{Introduction}

Hello World! \LaTeX. The euler formula is:
\[e^{i\pi} + 1 = 0\] 
and the riemann integral is:
\[ \int_{-\infty}^{\infty} \frac{1}{x^2 + 1} dx = \pi \]

\section{Algebra}
\lipsum[1]

\subsection{Linear algebra}
\lipsum[1]

\section{Example}
\lipsum[1]
\subsection{Example 1}

Sea $\{\tilde{\gamma}_{ij}\}_{0\leq i+j \leq 2n}$ una sucessión
de números reales tales que $\tilde{\gamma}_{00}>0$. Consideramos 
${\cal C}[t_1,t_2]$ el conjunto de polinomios complejos en $t_1$ y $t_2$
de grado $2n$ y definimos $\tilde{\varphi} : {\cal C} \left[t_1,t_2\right]
\longrightarrow \mathbb{C}$ y el funcional lineal complejo que interpola
a la sucessión $\tilde{\gamma}_{ij}$, es decir,
$$\tilde{\varphi}(t_1^it_2^j) = \tilde{\gamma}_{ij}$$.

\subsection{Example 2}

Una fórmula bien conocida es $\sum_{k=1}^{n}=\frac{n(n+1)}{2}$.

Otro ejemplo es $\displaystyle{\int_0^{2\pi} \sin x\ dx = 0}$.

Nótese la diferencia con:

$$\sum_{k=1}^{n}=\frac{n(n+1)}{2}$$

$$\textstyle{\int_0^{2\pi} \sin x\ dx = 0}$$

\subsection{Example 3}
La siguiente fracción es un montón de cosas:
$$\frac{\frac{a}{x-y}+\frac{b}{x+y}}{1+\frac{a-b}{a+b}}$$

\subsection{Example 4}
La siguiente matriz es una matriz:
$$
\left(
    \begin{array}{lll}
        a & b & c \\
        d & e & f \\
        g & h & i \\
    \end{array}
\right)
$$

En cambio la siguiente tiene puntos suspensivos:
$$
\left(
    \begin{array}{ccc}
        1      & \hdots & n \\
        2      & \hdots & n+1 \\
        \vdots & \ddots & \vdots \\
        n      & \hdots & 2n-1 \\
    \end{array}
\right)
$$

\subsection{Example 5}
La siguiente es una matriz:
$$\begin{matrix}
a & b & c \\ d & e & f \\ g & h & i
\end{matrix}$$
$$\begin{pmatrix}
a & b & c \\ d & e & f \\ g & h & i
\end{pmatrix}$$
$$\begin{bmatrix}
a & b & c \\ d & e & f \\ g & h & i
\end{bmatrix}$$
$$\begin{vmatrix}
a & b & c \\ d & e & f \\ g & h & i
\end{vmatrix}$$
$$\begin{Vmatrix}
a & b & c \\ d & e & f \\ g & h & i
\end{Vmatrix}$$

%% Euler equation of the second kind
\begin{equation}
\label{eq:euler}
e^{i\pi} + 1 = 0
\end{equation}

\subsection{Example 6}

The equation \eqref{eq:matrix} is a matrix. This matrix contains
a lot of things.
\begin{equation}
\label{eq:matrix}
%\tag{matrix}
\begin{matrix}
    a & b & c \\ d & e & f \\ g & h & i
\end{matrix}
\end{equation}



\end{document}